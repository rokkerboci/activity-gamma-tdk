%--------------------------------------------------------------------------------------
% TDK-specifikus változók
%--------------------------------------------------------------------------------------
\newcommand{\tdkszerzoB}{} % Második szerző neve; hagyd üresen, ha egyedül írtad a TDK-t.
\newcommand{\tdkev}{} % A dolgozat írásának éve (pl. "2014") (Ez OTDK-nál eltérhet az aktuális évtől.)

% További adatok az OTDK címlaphoz (BME-s TDK-hoz nem kell kitölteni)
\newcommand{\tdkevfolyamA}{IV} % Első szerző évfolyama, római számmal (pl. IV).
\newcommand{\tdkevfolyamB}{III} % Második szerző évfolyama, római számmal (pl. III).
\newcommand{\tdkkonzulensbeosztasA}{egyetemi tanár} % Első konzulens beosztása (pl. egyetemi docens)
\newcommand{\tdkkonzulensbeosztasB}{doktorandusz} % Második konzulens beosztása (pl. egyetemi docens)
