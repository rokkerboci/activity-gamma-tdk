%----------------------------------------------------------------------------
\section{Language Design}
%----------------------------------------------------------------------------

Every language starts with defining exactly what we want our language to accomplish. With the Activity Language, this was the following: it should support as much features from SysML activity diagrams as possible, all the while being simple to implement and transform to low-level models. As you could see from the \autoref{ssec:activities-as-petri-nets}, SysML activity diagrams cannot be exactly transformed to a formal model; we have to constrain our language in order to offer formal semantics. However, as you will see, these constrains still allow us to write \emph{almost} all activities.

\subsection{Constrains}

Compared to SysML activity diagrams, Activity Language has the following constrains:

\begin{itemize}
	\item No \emph{flow final} nodes.
	\item No \emph{receive signal} nodes.
	\item No \emph{interrupt} flows.
	\item Flows may only contain \emph{one} token at a given time.
	\item Activities can only be inside \emph{statecharts}, cannot describe a component's behaviour in itself
\end{itemize}

As you can see, most of these constrains can be substituted with other constructs, and just serve as \emph{shorthands}. The absence of \emph{flow final} nodes can be resolved by merging all flows into one single \emph{activity final} node. These steps could be done automatically using model transformation from SysML to Activity.

\subsection{Improvement over Petri net mapping}

Compared to the Petri net mapping introduced in \autoref{ssec:activities-as-petri-nets}, the Activity Language supports guards, valued tokens, actions and composite activities.
