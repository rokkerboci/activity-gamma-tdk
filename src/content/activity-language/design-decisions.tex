%----------------------------------------------------------------------------
\section{Language Design}
%----------------------------------------------------------------------------

The purpose of the Gamma Activity Language is the following: it should support as many features from SysML activity diagrams as possible, while also being simple to implement and transform to low-level models. In order to make the transformation easier, the language only supports a constrained subset of the SysML feature set, however, this does not hinder it's expressiveness.

\subsection{SysML Feature Subset}

Compared to SysML activity diagrams, GATL supports the following language constructs:

\begin{itemize}
	\item \emph{control} and \emph{data} flows.
	\item \emph{Initial} and \emph{activity final} nodes.
	\item \emph{Decision} and \emph{merge} nodes - without probability.
	\item \emph{Fork} and \emph{join} nodes.
	\item \emph{Action} nodes - which can contain inner activities (Call behaviour in SysML).
	\item \emph{Pins} on action nodes.
	\item \emph{Action} nodes - action nodes may contain specific internal actions\footnote{Written in the Gamma Action Language.}, that can calculate values, and send signals through specific ports.
\end{itemize}

\subsection{Constraints}\label{ssec:activity-constraints}

\todo{describe constraints in more detail}

\begin{itemize}
	\item only do action, no call behaviour on transitions
	\item activities cannot describe a component in themselves.
\end{itemize}
