%----------------------------------------------------------------------------
\section{Language Design}
%----------------------------------------------------------------------------

The purpose of the Gamma Activity Language is the following: it should support as many features from SysML activity diagrams as possible, while also being simple to implement and transform to low-level models. \todo{Ezt a részt át kéne írni} As you could see from the \autoref{ssec:activities-as-petri-nets}, SysML activity diagrams cannot be exactly transformed to a formal model; we have to constrain our language in order to offer formal semantics. However, as you will see, these constraints still allow us to write \emph{almost} all activities.

\subsection{Constraints}

Compared to SysML activity diagrams, Activity Language has the following constrains:

\begin{itemize}
	\item The model cannot contain \emph{flow final} nodes.
	\item The model cannot contain \emph{receive signal} nodes.
	\item The model cannot contain\emph{interrupt} flows.
	\item Flows may contain only \emph{one} token at a given time.
	\item Activities must be contained by \emph{statecharts}, as they cannot describe a component's behaviour in themself.
\end{itemize}

\subsection{Improvement over Petri Net Mapping}

Compared to the Petri net mapping introduced in \autoref{ssec:activities-as-petri-nets}, the Activity Language supports guards, valued tokens, actions and composite activities.
