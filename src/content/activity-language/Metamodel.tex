%----------------------------------------------------------------------------
\clearpage\section{Formal modelling}
%----------------------------------------------------------------------------


formal modeling howto, why needed, etc

semantic part, formal model, enforcing well formedness, etc

Due to the complexity of the meta-model, I have split it into multiple parts for easier understanding.


In order to offer mathematical precision, formal verification methods require formally defined models with clear semantics. 

Activities are composed of activity nodes, 

nodes have tokens
nodes are idle -> running -> done
flows can have tokens
nodes put tokens onto flows when done
node take tokens from flows when idle
if a node contains a token, it is considered running.

In this section, I present the meta-model and the formal semantics of the Gamma Activity Language.

\subsection{Formal Definition}

\begin{definition}[Gamma Activity Language]
	A Gamma Activity Language model is a tuple of \(GAL = (N, F, G, V, P)\), where:
	
	\begin{itemize}
		\item \(N = { n_1, n_2, \dots, n_n } \) is a set of \emph{Nodes};
		\item \(F = CF \bigcup DF \), where \( CF = { f_1, f_2, \dots, f_n } \) is a set of \emph{Control Flows}, and \( DF = { f_1, f_2, \dots, f_n } \), such that \(f \in F : n_1 \neq n_2 \forall \{n_1, n_2\} \in N \times N\). This means, that a given \emph{Flow} has to be defined between two different nodes;
		\item \(G : F -> \{true, false\} \) is a function determining whether a given Flow is enabled;
		\item \(V : N \) is a function returning the value contained in a node;
		\item \(P : N \) is a function returning the set of pins a node has.
	\end{itemize}
	
\end{definition}

The state of the Activity is defined by the state of nodes \(SN : N \rightarrow \{ Idle, Running, Done \}\) and the state of flows \(SF : F \rightarrow \{ Empty, Full \} \). The behaviour of the system is described as follows. We define three functions \emph{F1}, \emph{F2} and \emph{F3} (defined down below), of which one is non-deterministicly selected and run for each node. Let us denote the set of input/output flows as \(inflows(n)\) and \(outflows(n)\), respectively.

\paragraph{F1} 

F1 function is enabled if and only if \( \exists n \in N : SN(n) = Idle, \forall f \in inflows(n) : SF(f) = Full \). When F1 is enabled for a node \(n\) it fires the following way: \( SN(n) := Running, \forall f \in inflow(n) SF(f) := Empty \). In words: when there is an \emph{Idle} node, for which all input flows are \emph{Full} it fires, setting all input flow's state to \emph{Empty} and the nodes state to \emph{Running}. 

\paragraph{F2} 

F2 function is enabled if and only if \( \exists n \in N : SN(n) = Running \). When F2 is enabled for a node \(n\) it fires the following way: \( SN(n) := Done \). In words, this means a given node can stochastically change states form \emph{Running} to \emph{Done}.

\paragraph{F3} 

F3 function is enabled if and only if \( \exists n \in N : SN(n) = Done, \forall f \in outflows(n) : SF(f) = Empty, G(f) = true \). When F3 is enabled for a node \(n\) it fires the following way: \( SN(n) := Idle, \forall f \in outflows(n) SF(f) = Full \). In words, this means when a given node finished running, and all of its outpuf flows are \emph{Empty}, it will change its state to \emph{Idle}, and all of its output flows to \emph{Full} - transfering the token "state" from the node to the output flows.

komment: ezek a függvények felül vannak definiálva call-activity-re, decision node-ra és merge node-ra. Hogy írjam ezt le: sorban menjek végig a különböző fajta node-okon formálisan, vagy elég, ha ezt a metamodel leírásánál írom? Ezen felül még a pinek definícióját is be kéne vezetni: az ötletem az, hogy amikor egy data flow van, akkor egy adott pin-hez is hozzá van csatolva, és minden node-hoz tartozik egy "tároló" pin, ami egy adott értéket tárol. Amikor a data flow-ban átmegy a token, akkor innen olvas, vagy ide ír (és egy adott data flow csak egy tokent tud tárolni). Vagy, akár ezt úgy is lehetne, hogy a függvényeket úgy definiálom, hogy ha data flow így: - ha control flow akkor így: -. (persze akkor be kell vezetni a cf és df fogalmát)

\subsection{Root Structure}\label{ssec:root_structure}

Every model has to have a root element structure; Activity Language is not any different. The meta-model described in this section can be seen in \autoref{fig:declaration}.

\begin{figure}[!ht]
	\centering
	\includesvg[inkscapelatex=false, width=120mm, keepaspectratio]{declaration}
	\caption{The root structure of the language}
	\label{fig:declaration}
\end{figure}

\paragraph{Activity Declaration}\label{par:activity_declaration}

All elements inside an activity are contained in a root \emph{ActivityDeclaration} element. It contains \emph{Pins} needed for value passing (see \autoref{ssec:pins}) and a \emph{Definition}. A declaration can be \emph{InlineActivityDeclaration}, which means they are declared in an other declaration, or \emph{NamedActivityDeclaration}, which is a standalone activity declaration. The difference will be clarified in \autoref{ssec:composing_activities}.

\paragraph{Definition}\label{par:definition}

The definition \emph{defines} how the activity is described; using activity nodes, or by the Gamma Activity Language\footnote{Gamma Activity Language is a lightweight programming language-like construct for writing simple algorithms}. \emph{ActionDefinition} contains a single \emph{Block}\footnote{A \emph{Block} contains multiple \emph{Actions} which are executed one after the other}, which is executed as-is when the activity is executed\footnote{This fact means, that if one used Action to define an activity, that activity is executed atomically; it will not be interlaced with other XSTS transitions. See \autoref{ch:activiy_verification}.}. \emph{ActivityDefinition} contains \emph{ActivityNodes} (\autoref{ssec:activity_nodes}) and \emph{Flows} (\autoref{ssec:flows}).

\subsection{Activity Nodes}\label{ssec:activity_nodes}

In the following, I will talk about the different kinds of \emph{ActivityNodes} and their special meanings. All nodes are considered \emph{LAS} nodes by default. The meta-model described in this section can be seen in \autoref{fig:activity_nodes}.

\begin{figure}[!ht]
	\centering
	\includesvg[inkscapelatex=false, width=120mm, keepaspectratio]{activity_nodes}
	\caption{The Activity Node structure}
	\label{fig:activity_nodes}
\end{figure}

\paragraph{Pseudo Activity Node}

\emph{PseudoActivityNodes} are nodes, that do not represent a specific action, however are needed to convey specific meanings, e.g., the initial active node, or a decision between flows.

\paragraph{Initial Node}

\emph{InitialNodes} have one token in them when the containing activity is started. They shall only have (one or more) outgoing flows, and no input flows. 

\paragraph{Final Node}

When an activities \emph{FinalNode} gets a token the containing activity is considered \emph{Done}; after which the activity does not process any more tokens.

\paragraph{Data Node}

\emph{DataNodes} encapsulate the meaning of \emph{data} inside activities. A token may contain data (or value) of any kind, but that token can only travel to and from data \emph{sources} and \emph{targets}. More about this in \autoref{par:data_flow}.

\paragraph{Fork Node}

\emph{ForkNodes} are used to model parallelism, by creating one token on each of its output flows when executed. Fork nodes shall only have one input flow.

\paragraph{Join Node}

\emph{JoinNodes} are the pair of fork nodes; the additional created tokens are swallowed by this node, by only sending out one token, regardless of the number of input flows.

\paragraph{Decision Node}

\emph{DecisionNodes} create branches across multiple output flows. An input flows token is removed, and sent out to one, and only one of its output flows - depending on which of the output flows are \emph{enabled} (see \autoref{ssec:flows}).

\paragraph{Merge Node}

\emph{MergeNodes} 

\subsection{Composing Activities}

Lorem ipsum dolor sit amet, consectetur adipiscing elit. Ut vehicula turpis eget enim maximus, vel rutrum dui ullamcorper. Nulla enim ex, dapibus non aliquam vitae, molestie quis magna. Maecenas mattis turpis non ex feugiat, vitae pulvinar nisl vulputate.

\begin{figure}[!ht]
\centering
\includesvg[inkscapelatex=false, width=90mm, keepaspectratio]{composite-activity}
\caption{The }
\label{fig:composite_activity}
\end{figure}

\subsection{Flows}\label{ssec:flows}

Lorem ipsum dolor sit amet, consectetur adipiscing elit. Ut vehicula turpis eget enim maximus, vel rutrum dui ullamcorper. Nulla enim ex, dapibus non aliquam vitae, molestie quis magna. Maecenas mattis turpis non ex feugiat, vitae pulvinar nisl vulputate.

\paragraph{Control Flow}

Lorem ipsum dolor sit amet, consectetur adipiscing elit. Ut vehicula turpis eget enim maximus, vel rutrum dui ullamcorper. Nulla enim ex, dapibus non aliquam vitae, molestie quis magna. Maecenas mattis turpis non ex feugiat, vitae pulvinar nisl vulputate.

\paragraph{Data Flow}\label{par:data_flow}

Lorem ipsum dolor sit amet, consectetur adipiscing elit. Ut vehicula turpis eget enim maximus, vel rutrum dui ullamcorper. Nulla enim ex, dapibus non aliquam vitae, molestie quis magna. Maecenas mattis turpis non ex feugiat, vitae pulvinar nisl vulputate.

\begin{figure}[!ht]
	\centering
	\includesvg[inkscapelatex=false, width=90mm, keepaspectratio]{flows}
	\caption{The Flows structure}
	\label{fig:flows}
\end{figure}

\subsection{Pins}\label{ssec:pins}

Lorem ipsum dolor sit amet, consectetur adipiscing elit. Ut vehicula turpis eget enim maximus, vel rutrum dui ullamcorper. Nulla enim ex, dapibus non aliquam vitae, molestie quis magna. Maecenas mattis turpis non ex feugiat, vitae pulvinar nisl vulputate.

\paragraph{Input Pin}

Lorem ipsum dolor sit amet, consectetur adipiscing elit. Ut vehicula turpis eget enim maximus, vel rutrum dui ullamcorper. Nulla enim ex, dapibus non aliquam vitae, molestie quis magna. Maecenas mattis turpis non ex feugiat, vitae pulvinar nisl vulputate.

\paragraph{Output Pin}

Lorem ipsum dolor sit amet, consectetur adipiscing elit. Ut vehicula turpis eget enim maximus, vel rutrum dui ullamcorper. Nulla enim ex, dapibus non aliquam vitae, molestie quis magna. Maecenas mattis turpis non ex feugiat, vitae pulvinar nisl vulputate.

\begin{figure}[!ht]
\centering
\includesvg[inkscapelatex=false, width=60mm, keepaspectratio]{pins}
\caption{The Pins structure}
\label{fig:pins}
\end{figure}

\subsection{Data Source-Target Reference}

Lorem ipsum dolor sit amet, consectetur adipiscing elit. Ut vehicula turpis eget enim maximus, vel rutrum dui ullamcorper. Nulla enim ex, dapibus non aliquam vitae, molestie quis magna. Maecenas mattis turpis non ex feugiat, vitae pulvinar nisl vulputate.

\begin{figure}[!ht]
	\centering
	\includesvg[inkscapelatex=false, width=110mm, keepaspectratio]{data-source-target}
	\caption{The Data node reference structure}
	\label{fig:data_source_target_reference}
\end{figure}

\subsection{Pin Reference}

Lorem ipsum dolor sit amet, consectetur adipiscing elit. Ut vehicula turpis eget enim maximus, vel rutrum dui ullamcorper. Nulla enim ex, dapibus non aliquam vitae, molestie quis magna. Maecenas mattis turpis non ex feugiat, vitae pulvinar nisl vulputate.

\begin{figure}[!ht]
	\centering
	\includesvg[inkscapelatex=false, width=110mm, keepaspectratio]{pin-reference}
	\caption{The Data node reference structure}
	\label{fig:pin_reference}
\end{figure}
