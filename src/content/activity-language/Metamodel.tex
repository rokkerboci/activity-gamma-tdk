%----------------------------------------------------------------------------
\clearpage\section{Metamodel}
%----------------------------------------------------------------------------

The metamodel of the language was constructed to facilitate well defined semantics, but have as few model elements as possible.

\subsection{Semantic part}

This part provides elements needed to construct a semantically correct model. An Activity model has \textbf{Nodes} and \textbf{Flows} in between nodes. Flows can be connected to nodes directly, or to \textbf{Pins}, which gives us explicit control over \emph{control} flows and \emph{data} flows. See figure \ref{fig:TeXstudio}

\begin{figure}[!ht]
	\centering
	\includesvg[inkscapelatex=false, width=100mm, keepaspectratio]{semantics}
	\caption{Using Gamma to verify high-level models}
	\label{fig:semantic}
\end{figure}

\subsection{Syntactic part}

The metamodel also has to provide elements for the language, such as \textbf{Declaration}s, \textbf{Definition}s, etc.

\begin{figure}[!ht]
	\centering
	\includesvg[inkscapelatex=false, width=100mm, keepaspectratio]{semantics}
	\caption{Using Gamma to verify high-level models}
	\label{fig:syntactic}
\end{figure}
