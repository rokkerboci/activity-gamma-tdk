%----------------------------------------------------------------------------
\clearpage\section{Language Gramar}\label{sec:activity-grammar}
%----------------------------------------------------------------------------

Similarly to the Gamma Statechart Language, the Gamma Activity Language is intended to be a first-class citizen in the Gamma Framework, thus it must have a grammar to represent in a textual way. This implementation incorporates elements from the SysMLv2 \cite{omg_sysml_v2} language design, while also fitting into the already existing language family of Gamma (\autoref{sec:gamma}). In the following, I define the \emph{metamodel} and the \emph{grammar} created to implement the formalism specified in \autoref{sec:activity-formal-definition}.

\subsection{Metamodel}

Due to the complexity of the final metamodel of the language, I have split it into multiple parts for easier understanding: Pins (\autoref{ssec:pins}), Flows (\autoref{ssec:flows}), Activity Nodes  (\autoref{ssec:activity_nodes}), Structure (\autoref{ssec:activity_nodes}), Composite Activities (\autoref{ssec:composing-activities}) and Data-, Pin-reference (\autoref{ssec:data-reference}-\autoref{ssec:pin-reference}).

\subsubsection*{Pins}\label{ssec:pins}

Pins are categorised into two sets, \emph{InputPins} and \emph{OutputPins}. All pins have a \emph{Type}\footnote{Types come from the Gamma Expression Language.} associated with them, which define their domain. \autoref{fig:pins} shows this section of the metamodel.

\begin{figure}[!ht]
	\centering
	\includesvg[inkscapelatex=false, width=60mm, keepaspectratio]{pins}
	\caption{The pin metamodel diagram}
	\label{fig:pins}
\end{figure}

\subsubsection*{Flows}\label{ssec:flows}

Flows have two kinds, \emph{ControlFlows} and \emph{DataFlows}. All flows have a guard of type \emph{Expression}, which can evaluate to a \emph{boolean} value; if it evaluates to \emph{True}, the flow is considered \emph{enabled}. \autoref{fig:flows} depicts the relation of flows in the metamodel.

\begin{figure}[!ht]
	\centering
	\includesvg[inkscapelatex=false, width=90mm, keepaspectratio]{flows}
	\caption{The Flows structure}
	\label{fig:flows}
\end{figure}

\paragraph{Control Flow}

\emph{ControlFlows} have a reference to their source and target nodes.

\paragraph{Data Flow}\label{par:data_flow}

\emph{DataFlows} contain a \emph{DataSourceReference} and a \emph{DataTargetReference}, which can either be a \emph{Pin}, or a \emph{DataNode}. A data token may contain a value of any kind, but that token can travel only to and from data \emph{sources} and \emph{targets}. This will be explained in more detail below in \autoref{ssec:data-reference}.

\subsubsection*{Activity Nodes}\label{ssec:activity_nodes}

In the following, I will talk about the different kinds of \emph{ActivityNodes} and their special meanings. The metamodel described in this section can be seen in \autoref{fig:activity_nodes}.

\begin{figure}[!ht]
	\centering
	\includesvg[inkscapelatex=false, width=120mm, keepaspectratio]{activity_nodes}
	\caption{The Activity Node structure}
	\label{fig:activity_nodes}
\end{figure}

\paragraph{Action Node}

\emph{ActionNodes} represent a specific action the activity may execute. This action can be defined in multiple ways, see \autoref{ssec:composing-activities} for more detail.

\paragraph{Pseudo Activity Node}

\emph{PseudoActivityNodes} are nodes, that do not represent a specific action, however are needed to convey specific meanings, e.g., the initial active node, or a decision between flows.

\paragraph{Initial Node}

\emph{InitialNodes} represent the entry point of the activity.

\paragraph{Final Node}

A special node representing the final node of the activity.

\paragraph{Data Node}

\emph{DataNodes} encapsulate the meaning of \emph{data} inside activities.

\paragraph{Fork Node}

\emph{ForkNodes} are used to model parallelism, by creating one token on each of its output flows when executed. Fork nodes shall have only one input flow.

\paragraph{Join Node}

\emph{JoinNodes} are the complementary elements of fork nodes; the additional created tokens are swallowed by this node, by sending out one token, regardless of the number of input flows.

\paragraph{Decision Node}

\emph{DecisionNodes} create branches across multiple output flows. An input flow's token is removed, and sent out to a single output flow - depending on which of the output flows are \emph{enabled} (see \autoref{ssec:flows}).

\paragraph{Merge Node}

\emph{MergeNodes} forward all incoming tokens as soon as they arrive, one by one. They are used to \emph{merge} different flow paths (created using \emph{decisions}).

\subsubsection*{Root Structure}\label{ssec:root_structure}

The root structure defines where the elements of the activity reside in the model. The metamodel described in this section can be seen in \autoref{fig:declaration}.

\begin{figure}[!ht]
	\centering
	\includesvg[inkscapelatex=false, width=100mm, keepaspectratio]{declaration}
	\caption{The root structure of the language}
	\label{fig:declaration}
\end{figure}

\paragraph{Activity Declaration}\label{par:activity_declaration}

All elements inside an activity are contained in a root \emph{ActivityDeclaration} element. It contains \emph{Pins} needed for value passing (see \autoref{ssec:pins}) and a \emph{Definition}. Declarations can be \emph{InlineActivityDeclarations}, which means they are declared in an other declaration, or \emph{NamedActivityDeclaration}, which is a standalone activity declaration - \emph{NamedActivityDeclaration} can be referenced from other parts of the model, while \emph{InlineActivityDeclaration} cannot.

\paragraph{Definition}\label{par:definition}

Definitions define how the activity is described; using activity nodes, or by the Gamma Action Language\footnote{Gamma Action Language is a lightweight action language, created to describe actions on statechart transitions.}. \emph{ActionDefinition} contains a single \emph{Block}\footnote{A \emph{Block} contains multiple \emph{Actions} which are executed one after the other}, which is executed as is when the activity is executed\footnote{This means, that upon execution the given activity is executed atomically; it will not be interlaced with other XSTS transitions. See \autoref{ssec:activity-constraints}.}. \emph{ActivityDefinitions} contain \emph{ActivityNodes} (\autoref{ssec:activity_nodes}) and \emph{Flows} (\autoref{ssec:flows}), and are used to \emph{compose} activities (see \autoref{ssec:composing-activities}).

\subsubsection*{Composing Activities}\label{ssec:composing-activities}

\emph{ActionNodes} may contain a single \emph{ActivityDeclarationReference} to an \emph{ActivityDeclaration}\footnote{If the node does not have any, it is considered a \emph{simple} node, without any implementation.}, in which case the execution of said node will also include the execution of the underlying \emph{ActivityDeclaration}. This construction gives us the power to pre-declare, or inline specific activities in the model; and use any definition defined above (\autoref{ssec:root_structure})  for them. \autoref{fig:composite_activity} shows this part of the metamodel.

\begin{figure}[!ht]
	\centering
	\includesvg[inkscapelatex=false, width=80mm, keepaspectratio]{composite-activity}
	\caption{The action node containment hierarchy}
	\label{fig:composite_activity}
\end{figure}

\subsubsection*{Data Source-Target Reference}\label{ssec:data-reference}

In order to correctly set a data flow's data source and data target, we have to keep in mind where we are referencing the pins. A given \emph{InputPin} can be considered a \emph{DataTarget} from outside of the associated \emph{Activity}, however, it is a \emph{DataSource} from inside the \emph{Defintion}. \emph{DataNodes} can be considered both data sources and data targets. See \autoref{fig:data_source_target_reference} for the metamodel.

\begin{figure}[!ht]
	\centering
	\includesvg[inkscapelatex=false, width=100mm, keepaspectratio]{data-source-target}
	\caption{The Data Source-Target reference structure}
	\label{fig:data_source_target_reference}
\end{figure}

\paragraph{Example of inside-input/outside-output pin} \autoref{fig:data_source_target_reference_example} shows an example for the multi directional effect of pins. From the perspective of the flow \(P_1 \rightarrow P_2\) the pin \(P_1\) is a \emph{DataSource} and the pin \(P_2\) is a \emph{DataTarget}. However, from the perspective of the flow \(P_2 \rightarrow P_3\), the pin \(P_2\) is a \emph{DataSource} - because the latter flow is inside the composite activity.

\begin{figure}[!ht]
	\centering
	\includesvg[inkscapelatex=false, width=60mm, keepaspectratio]{data-source-target-example}
	\caption{The Data node reference structure}
	\label{fig:data_source_target_reference_example}
\end{figure}

\subsubsection*{Pin Reference}\label{ssec:pin-reference}

Pin references are used to define a rigid pin-reference structure in the model. \emph{InputPinReferences} have a reference to a specific \emph{InputPin}, \emph{OutputPinReferences} have a reference to a specific \emph{OutputPin}. \emph{InsidePinReferences} and \emph{OutsidePinReferences} are used to define the direction in which the reference sees the given pin; inside references see it from the inside, outside references see it from the outside. The \emph{OutsidePinReference} must also have a reference to the specific \emph{ActionNode} the pin is associated with\footnote{Note, that the inside pin reference does not have a node reference, because the node must be the activity that contains the pin.}. See \autoref{fig:pin_reference} for the metamodel.

\begin{figure}[!ht]
	\centering
	\includesvg[inkscapelatex=false, width=100mm, keepaspectratio]{pin-reference}
	\caption{The Pin reference structure}
	\label{fig:pin_reference}
\end{figure}

\subsection{Concrete Syntax}

In order to make it easier to test and visualise activities, I created an Xtext domain-specific language. The SysMLv2 language served as the main inspiration for the language, however, much of the \emph{syntactic sugars} have been omitted for simplicities' sake. As a result, many constructs are inherently repetitive to write. You can see an example in \autoref{lst:gamma-activity}. For the exact language definition, please see \autoref{sec:gamma-activity-language-constructs}

\begin{lstlisting}[float,language=activity, caption={Gamma Activity Language representation of the compilation activity.}, label={lst:gamma-activity}]
activity CompilationProcess {
	var errors : boolean := false
	
	initial Initial
	merge Merge
	fork Fork
	
	action Read1 : activity(
		out p : integer
	)
	action Compile1 : activity(
		in p : integer
	)
	action Read2 : activity(
		out p : integer
	)
	action Compile2 : activity(
		in p : integer
	)
	
	join Join
	decision Decision
	action Edit
	final Final
	
	control flow from Initial to Merge 
	// ...	
	data flow from Read1.p to Compile1.p
	control flow from Fork to Read2 
	// ...
	control flow from Decision to Edit [errors]
	control flow from Edit to Merge [!errors]
	control flow from Decision to Final
}	
\end{lstlisting}