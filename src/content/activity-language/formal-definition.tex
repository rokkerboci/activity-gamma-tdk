%----------------------------------------------------------------------------
\section{Formal Definition}\label{sec:activity-formal-definition}
%----------------------------------------------------------------------------

\newcommand{\inflows}[1]{\delta(#1)}
\newcommand{\outflows}[1]{\Delta(#1)}
\newcommand{\sourcepin}[1]{\phi(#1)}
\newcommand{\targetpin}[1]{\Phi(#1)}
\newcommand{\sourcenode}[1]{\theta(#1)}
\newcommand{\targetnode}[1]{\Theta(#1)}

In order to offer mathematical precision, formal verification methods require formally defined models with clear semantics. In this section I present the formal definition of the novel GATL formalism.

\begin{definition}[Gamma Activity Language]
	A Gamma Activity is a tuple of \(\mathit{GATL} = (N, P, F, G, D_\mathit{GA}, F_{Action})\), where:
	
	\begin{itemize}
		\item \(N = N_\mathit{IR} \bigcup N_\mathit{LAS} \) is a set of \emph{Nodes}, where \(N_\mathit{IR}\) contains the \emph{immediate-repeat} nodes and \(N_\mathit{LAS}\) contains the \emph{load-and-sand} nodes (see \autoref{ssec:activities-as-petri-nets}). \(N_\mathit{Action} \in N_\mathit{LAS} \) is a special set of nodes, which are the \emph{Action} nodes;
		\item \( P \subseteq P_\mathit{In} \bigcup P_\mathit{Out} \) is a set of two types of pins, where \(P_\mathit{In} : N_\mathit{Action} \rightarrow \{ p_1, \dots, p_n \} \) and \(P_\mathit{Out} : N_\mathit{Action} \rightarrow \{ p_1, \dots, p_n \} \) are the set of \emph{InputPins} and \emph{OutputPins}, respectively, with domains \(\{ d_1, \dots, d_n \} \subseteq D_\mathit{GA} \);
		\item \( F \subseteq F_C \bigcup F_D \), where \(F_C = \{ f_{C_1}, \dots, f_{C_n} \} \) and \(F_D = \{ f_{D_1}, \dots, f_{D_n} \} \) are the control and data flows, respectively. Let us denote the input/output flows of node \(n\) as \( \inflows{n} \) and \( \outflows{n} \), and the source/target pins of flow \(f\) as \( \sourcepin{f} \) and \(\targetpin{f}\). For any given node \(n\), \( \inflows{n} \neq \outflows{n} \) and for any given action node \(n_a\) \( \targetpin{f} \in P_\mathit{In}(n_a) \forall f \in F_D \bigcap \inflows{n_a} \) and \( \sourcepin{f} \in P_\mathit{Out}(n_a) \forall f \in F_D \bigcap \outflows{n_a} \) shall always hold. This means, that a flow cannot be input and output to the same node at the same time, and for a given action node, all input/output flows shall be associated with an input/output token, respectively;
		\item \(G : F \rightarrow \{ \mathit{True}, \mathit{False} \} \) is a function determining whether a given flow is enabled;
		\item \(D_\mathit{GA} = \{ d_{1}, d_{2}, \dots, d_{n} \} \) is a set of value domains;
		\item \(F_{Action} \subseteq d_{1} \times d_{2} \times \dots \times d_{n} \) is a function mapping an \emph{action} nodes input pin values to it's output pin values.
	\end{itemize}
	
Informally, Gamma Activities are composed of \emph{nodes} and flows in between them. A given  \emph{Action} node may have any number of \emph{Pins} with domain \(d \in D\), for which, there must be one and only one flow connected to the node. Nodes \(n \in N_\mathit{Action}\) may also contain internal activities denoted \(\mathit{GA}_\mathit{n}\).

\(N_\mathit{IR}\) contains the nodes \emph{decision/merge}, while \(N_\mathit{LAS}\) contains the nodes \emph{fork/join}, \emph{initial/final} and \emph{action} nodes. For the most part, these elements have a similar behaviour as their SysML counter parts; however, there is a crucial difference regarding how a flow transmits a token. \autoref{fig:activity-data-flow1} shows a simple data flow between two nodes, connected via their pins. When this flow is fired, the data inside the pins are transferred instantly, without any intermediate state in between. 

\begin{figure}[!ht]
	\centering
	\includesvg[inkscapelatex=false, width=60mm, keepaspectratio]{activity-data-flow1}
	\caption{An example SysML data flow}
	\label{fig:activity-data-flow1}
\end{figure}

Contrarily, in Gamma Activities, the equivalent data flow does indeed contain the given node, creating an intermediate state, where the token is in neither of the nodes. This would look like a \emph{Central Buffer} in SysML (\autoref{fig:activity-data-flow2}). The reason for this is simplicity; by creating this intermediate state (and others), it is easier to state the set of transitions necessary to formally define the semantics of the language.

\begin{figure}[!ht]
	\centering
	\includesvg[inkscapelatex=false, width=70mm, keepaspectratio]{activity-data-flow2}
	\caption{An example SysML data flow with central buffer}
	\label{fig:activity-data-flow2}
\end{figure}
	
\end{definition}\label{def:activity-structure}

\subsection{Behaviour Formalism}

The behaviour of GATL is defined using the XSTS (\autoref{sec:xsts}) formalism:

\begin{definition}[Gamma Activity Behaviour]
	\( \mathit{XSTS_{GA}} = (D, V, V_C, \mathit{IV}, \mathit{Tr}, \mathit{In}, \mathit{En}) \), where:
	
	\begin{itemize}
		\item \( D = D_\mathit{GA} \bigcup d_\mathit{NodeState} \bigcup d_\mathit{FlowState} \), where \(D_\mathit{GA}\) is the domain in \emph{GA}, \(d_\mathit{NodeState} = \{ \mathit{Idle}, \mathit{Running}, \mathit{Done} \} \) and \(d_\mathit{FlowState} = \{ \mathit{Empty}, \mathit{Full} \} \) are the node and flow state domains representing their internal state. Node state \emph{Idle} means a node is ready for a token, \emph{Running} means it is currently executing and \emph{Done} means it is done (but still contains a token). Flow states \emph{Empty} and \emph{Full} represent whether the flow contains a token;
		\item \( V = V_\mathit{NS} \bigcup V_\mathit{FS} \bigcup V_\mathit{NV} \bigcup V_\mathit{FV} \bigcup V_\mathit{PV} \), where \(V_\mathit{NS} = \{ v_1, v_2, \dots, v_n \} \) is the variable representing the \emph{state} of the nodes with domain \(d_\mathit{NodeState}\), \(V_\mathit{FS} = \{ v_1, v_2, \dots, v_n \} \) is the variable representing the \emph{state} of the flows with domain \(d_\mathit{FlowState}\) and \( V_\mathit{NV} \), \( V_\mathit{FV} \), \( V_\mathit{PV} \) are the variables representing the values contained inside \emph{flows}, \emph{nodes} and \emph{pins} respectively;
		\item \( IV \) sets all variables to their default value;
		\item \( V_C = V_\mathit{NS} \bigcup V_\mathit{FS} \) are the \emph{control variables}\footnote{We don't have to set the value variables as control variables, because they only change when the given flow's or node's state variable changes.};
		\item \( \mathit{Tr} = T \), where \(T\) is the single transition defined down below;
		\item \( \mathit{In} \) sets all values of variables associated with \emph{InitialNodes} to \emph{Running};
		\item \( \mathit{En} = \emptyset \) is the empty environment transition, as simple activities do not have environments.
	\end{itemize}
	
\end{definition}\label{def:activity-state}

We also define the following helper functions:

\begin{itemize}
	\item \(S_N : N \rightarrow V_\mathit{NS} \) is a function that returns the \emph{state variable} of a node;
	\item \(S_F : N \rightarrow V_\mathit{FS} \) is a function that returns the \emph{state variable} of a flow;
	\item \(V_{N_\mathit{In}} : N \times F \rightarrow V_\mathit{NV} \) is a function that returns the \emph{input value variable} of a node regarding a specific connected flow;
	\item \(V_N : N \rightarrow V_\mathit{NV} \) is a function that returns the \emph{value variable} of a node;
	\item \(V_F : F \rightarrow V_\mathit{FV} \) is a function that returns the \emph{value variable} of a flow;
	\item \(\mathit{PV}_N : N \times P \rightarrow V_\mathit{PV} \) is a function returning the \emph{value variable} of a pin inside a node.
\end{itemize}

Informally, the state of Gamma Activities are determined by the nodes' \emph{state} (\emph{Idle}, \emph{Running}, \emph{Done}), the nodes' and their pins' values, the flows' \emph{state} (\emph{Empty}, \emph{Full}) and their values. For example, given an action node \(n\) and one of it's pins \(p_1\), \(\mathit{PV}_N(n, p)\) would give us the exact value that pin contains in this instance. This gives us the power - contrary to the Activity-PN mapping introduced in \autoref{ssec:activities-as-petri-nets} - to formally define the values contained in tokens.

This definition gives us the structure to define the exact behaviour of a Gamma Activity. In the following, I incrementally build up the parts of transition \(T\), and then propose an algorithm constructing the XSTS. 

Let us define a shorthand function for finding a nodes input variable associated with a given flow:

\begin{equation*}
	V_{N_\mathit{In}}(n, f) = 
	\begin{cases}
		\mathit{PV}_N(n, \targetpin{f}) & \text{if } n \in N_{Action} \\
		V_N(n), & \text{otherwise}
	\end{cases}
\end{equation*}

And a shorthand function for finding a nodes output variable associated with a given flow:

\begin{equation*}
	V_{N_\mathit{Out}}(n, f) = 
	\begin{cases}
		\mathit{PV}_N(n, \sourcepin{f}) & \text{if } n \in N_{Action} \\
		V_N(n), & \text{otherwise}
	\end{cases}
\end{equation*}

These notations will help us easily transfer the data to and from the node value variables - by hiding the difference between the pin variables and the node variables.

\newcommand{\enabled}{\mathit{En}}

Next, I define various functions that return operations, which will be used to construct the transition \(T\). In the following, I will notate \emph{sequences} as operations after one another, \emph{choices} as \(\mathit{choice}(p)\), \emph{parallels} as \(\mathit{parallel}(p)\), where \(p\) is a set of operations, \emph{assumptions} as \(\mathit{assume}(v = value)\), where \(v\) is a varible, and \(v = value\) is an expression returning a boolean, and lastly, \emph{assignments} as \(\mathit{assign}(v, value)\).

\begin{definition}[Flow Transition Operations]	
	Let us denote the source/target node of flow \(f\) as \( \sourcenode{f} \) and \( \targetnode{f} \) respectively. \(O_\mathit{FlowIn}(f)\) is a function returning a \emph{sequential} operation, which takes a token from a flow to a node:
	
	\begin{align*}
		&\mathit{assume}(S_F(f) = \mathit{Full}) \\
		&\mathit{assume}(S_N(\targetnode{f}) = \mathit{Idle}) \\
		&\mathit{assign}(S_F(f), \mathit{Empty}) \\
		&\mathit{assign}(S_N(\targetnode{f}), \mathit{Running}) \\
		&\mathit{assign}(S_N(V_{N_\mathit{In}}(\targetnode{f}, f)), V_F(f)) \\
	\end{align*}

	Informally, the returned operation checks that the node is in \emph{Idle} state and the flow is in \emph{Full} state. After which, it transfers the token by changing their states. Let \(O_\mathit{FlowOut}(f)\) be the function returning a \emph{sequential} operation, which takes a token from a node to a flow:
	
	\begin{align*}
		&\mathit{assume}(G(f) = \mathit{true}) \\
		&\mathit{assume}(S_F(f) = \mathit{Empty}) \\
		&\mathit{assume}(S_N(\sourcenode{f}) = \mathit{Done}) \\
		&\mathit{assign}(S_F(f), \mathit{Full}) \\
		&\mathit{assign}(S_N(\sourcenode{f}), \mathit{Idle}) \\
		&\mathit{assign}(V_F(f), S_N(V_{N_\mathit{Out}}(\sourcenode{f}, f))) \\
	\end{align*}

	The returned operation checks that the node is \emph{Done} and the flow is \emph{Empty} and \emph{enabled}, in which case it transfers the token by changing their states. Note, that both of the operations transfer the value of the node to the flow, or the value of the flow to the node using the previously defined helper notations.
\end{definition}

\begin{definition}[Node Token In/Out Operations]
	Given the different behaviours of \emph{IR} and \emph{LAS} nodes, we must construct different operations for their in and out transitions. \(O_\mathit{IRNodeIn}(n)\) is the function returning a \emph{choice} operation that represents the \enquote{token intake} of an \emph{IR} node:
	
	\begin{equation*}
		O_\mathit{IRNodeIn}(n) = \mathit{choice}\left(\bigcup_{f \in \inflows{n}} O_\mathit{FlowIn}(f)\right)
	\end{equation*}
	
	And \(O_\mathit{IRNodeOut}(n)\) is the function returning a \emph{choice} operation that represents the \enquote{token output} of an \emph{IR} node:

	\begin{equation*}
		O_\mathit{IRNodeOut}(n) = \mathit{choice}\left(\bigcup_{f \in \outflows{n}} O_\mathit{FlowOut}(f)\right)
	\end{equation*}

	On a similar note, the functions constructing the \emph{LAS} node \enquote{token intake} and \enquote{token output} operations are the following:
	
	\begin{align*}
		O_\mathit{LASNodeIn}(n) &= \mathit{parallel}\left(\bigcup_{f \in \inflows{n}} O_\mathit{FlowIn}(f)\right)\\
		O_\mathit{LASNodeOut}(n) &= \mathit{parallel}\left(\bigcup_{f \in \outflows{n}} O_\mathit{FlowOut}(f)\right)
	\end{align*}

	For simplicities sake, I define the following functions to merge all the node in/out operations:

	\begin{align*}
		O_\mathit{LASNodeIO}(n) &= \mathit{choice}\left(O_\mathit{LASNodeIn}(n) \bigcup O_\mathit{LASNodeOut}(n)\right)\\
		O_\mathit{IRNodeIO}(n) &= \mathit{choice}\left(O_\mathit{IRNodeIn}(n) \bigcup O_\mathit{IRNodeOut}(n)\right)\\
		O_\mathit{NodeIO}(n) &= 
		\begin{cases}
			O_\mathit{IRNodeIO}(n) & \text{if } n \in N_{IR} \\
			O_\mathit{LASNodeIO}(n), & \text{otherwise}
		\end{cases}
	\end{align*}

	In which case, \(O_\mathit{NodeIO}(n)\) returns an operation transferring the tokens in and out of the specified node. 

\end{definition}

Informally, these operations realise the semantics of the \emph{LAS} and \emph{IR} nodes, by either putting the \emph{flow in} and \emph{flow out} operations inside a \emph{parallel} or a \emph{choice}, respectively\footnote{Parallel operations execute \emph{all} contained operations at the same time, while choices chose one at random.}.

\begin{definition}[Node Run Operations]
	Node transition operations take the given node from \emph{Running} state to \emph{Done} state, while also executing the underlying operation. Generally, \(O_\mathit{NodeRun}(n)\) function is defined as:
	
	\begin{align*}
		&\mathit{assume}(S_N(n), \mathit{Running}) \\
		&\mathit{assign}(S_N(n), \mathit{Done}) \\
	\end{align*}

	However, some nodes have special behaviours. If the node is an \emph{Action} node, it can contain either a \emph{Gamma Action} expression, or a composite \emph{Activity} (see \autoref{sec:activity-grammar} for more information).

	\begin{itemize}
		\item If the node is an \emph{Action} node, the contained \emph{Gamma action} is mapped to an operation \(O_\mathit{Action}(n)\), and added to \( O_\mathit{NodeRun}(n) \). Let us call this function \(O_\mathit{NodeRun}^\prime(n)\)
		\item If the node is an \emph{Action} node, the contained activity first has to be started, and the state of the node can only change when the contained activity is done. Let us call this function as \(O_\mathit{NodeRun}^{\prime\prime}(n)\).
	\end{itemize}

	\(O_\mathit{Run}^\prime(n)\) function is defined as:
	
	\begin{align*}
		&\mathit{assume}(S_N(n), \mathit{Running}) \\
		&O_\mathit{Action}(n) \\
		&\mathit{assign}(S_N(n), \mathit{Done}) \\
	\end{align*}

	Let \(\mathit{In}(n)\) and \(\mathit{Fin}(n)\) be the sets of \emph{initial} and \emph{final} nodes in \(\mathit{GA}_\mathit{n}\). 
	
	\begin{align*}
		O_\mathit{Start}(n) &= \mathit{sequence}\left(
			\bigcup_{n_\mathit{In} \in \mathit{In}(n)} \left(\mathit{assume}(S_N(n_\mathit{In}) = \mathit{Idle}) \bigcup \mathit{assign}(S_N(n_\mathit{In}), \mathit{Running})
		\right)\right)\\
		O_\mathit{Finish}(n) &= \mathit{sequence}\left(
			\bigcup_{n_\mathit{Fin} \in \mathit{Fin}(n)} \mathit{assume}(S_N(n_\mathit{Fin}) = \mathit{Done}) \bigcup \mathit{assign}(S_N(n), \mathit{Done})
		\right)
	\end{align*}

	\begin{equation*}
		O_\mathit{ActivityRun}(n) = \mathit{choice}\left(O_\mathit{Start}(n) \bigcup O_\mathit{Finish}(n)\right)
	\end{equation*}
	
	\begin{align*}
		O_\mathit{NodeRun}^{\prime\prime}(n) &= \mathit{sequence}\left(\mathit{assume}(S_N(n), \mathit{Running}) \bigcup O_\mathit{ActivityRun}(n)\right)
	\end{align*}

	Informally, the returned operation checks if the node is in state \emph{Running}, and then chooses: it either sets the contained activity's initial nodes \emph{Running}, if they are currently \emph{Idle}, or sets the node's state \emph{Done} if the contained activity's final nodes are \emph{Done}. Thus, the node is only considered \emph{Done}, when the contained activity is also done.

\end{definition}

\begin{figure}[!ht]
\centering
\begin{subfigure}[b]{\textwidth}
	\centering
	\includesvg[inkscapelatex=false, width=80mm, keepaspectratio]{activity-state-function}
	\caption{\(O_\mathit{Node}\) operations for a \emph{LAS} node}
	\label{fig:activity-state-function-las}
\end{subfigure}\\
\vspace{10mm}
\begin{subfigure}[b]{\textwidth}
	\centering
	\includesvg[inkscapelatex=false, width=80mm, keepaspectratio]{activity-state-function-ir}
	\caption{\(O_\mathit{Node}\) operations for an \emph{IR} node}
	\label{fig:activity-state-function-ir}
\end{subfigure}\\
\vspace{10mm}
\begin{subfigure}[b]{\textwidth}
	\centering
	\includesvg[inkscapelatex=false, width=60mm, keepaspectratio]{activity-state-function-composite}
	\caption{\(O_\mathit{NodeRun}\) operations for a \emph{composite} action node}
	\label{fig:activity-state-function-composite}
\end{subfigure}\\
\caption{An illustration of the state change operations}
\label{fig:activity-state-functions}
\end{figure}

\begin{definition}[Composite Node Operation]
	Putting all of the work together, we can define a function returning the operation representing all behaviour of a given node:
	
	\begin{equation*}
		O_\mathit{Node}(n) = \mathit{choice}\left(O_\mathit{NodeIO}(n) \bigcup O_\mathit{NodeRun}(n)\right)
	\end{equation*}
\end{definition}

Informally, this is a choice operation, that either moves the tokens in/out, or executes the internal action of the node. The various operations defined in the previous sections can be seen in \autoref{fig:activity-state-functions}. 

\iffalse

\begin{definition}
	
\begin{itemize}
	\item Transition \( T_\mathit{In} \) \enquote{takes} the token(s) from the incoming flow(s), and puts them/it onto the selected node \(n \in N\). The operation of the transition depends on the type of the selected node:
	\begin{itemize}
		\item If the selected node \(n \in N\) is an \emph{IR} node (\(n \in N_\mathit{IR}\)), then the operation is a \emph{choice} operation, containing the following \emph{sequential} operations: \( \emph{seq}_n = \{ assume(S_F(f_n) = \mathit{Full} \wedge S_N(n) = \mathit{Idle});  ) \} \) \( T_\mathit{In} \) is a \emph{choice} upon execution one contained transition is selected and executed nondeterministically.
		\item If the selected node \(n \in N\) is a \emph{LAS} node (\(n \in N_\mathit{LAS}\)), upon execution all contained transitions are executed.
		\item If \(n \in N_\mathit{IR} \), the enabling function \( \enabled(T_\mathit{In}) \) returns \( \exists f \in \inflows{n} : S_F(f) = \mathit{Full} \). When the transition is executed, a flow is selected arbitrary: \(f f \in \inflows{n} : S_F(f) = \mathit{Full} \) and the next states become: \( S_F^\prime(f) = \mathit{Empty} \), \( S_N^\prime(n) = \mathit{Running} \).
	\end{itemize}
	\item Transition \( T_\mathit{Run} \) \enquote{runs} the operation associated with the given node, after which the next state is defined as: \(S_N^\prime(n) = \mathit{Done}\). 
\end{itemize}

leírom a state transition function-t, amely egy állapotból a következőbe visz. Utána ir-las node-ok-hoz a különböző szemantikák (fin 2, fout 2, frun 1), valamint a token átviteli függvény (control flow vs data flow). itt majd vissza kell utalni a background-ba, úgyhogy előbb azt kell megcsinálni.

\begin{equation}
	F_{In} : N \rightarrow
	\begin{cases}
		\begin{aligned}[b]
			&\text{select } f_s f_s \in \inflows{n} \wedge S_F(f) = \mathit{Full} \\
			&V_N^\prime(n) = V_F(f_s) \\
			&S_F^\prime(f_s) = \mathit{Empty} \\
			&S_N^\prime(n) = \mathit{Running},
		\end{aligned} & \text{if } n \in N_\mathit{Merge} \\[8pt]
		\begin{aligned}[b]
			&\mathit{PV}_N^\prime(n, \targetpin{f}) = V_F(f) : \forall f \in \inflows{n} \bigcap F_D \\
			&S_F^\prime(f) = \mathit{Empty} : \forall f \in \inflows{n} \\
			&S_N^\prime(n) = \mathit{Running},
		\end{aligned} & \text{if } n \in N_\mathit{Action} \\[8pt]
		\begin{aligned}[b]
			&V_N^\prime(n) = V_F(f) : f \in \inflows{n} \bigcap F_D \\
			&S_F^\prime(f) = \mathit{Empty} : \forall f \in \inflows{n} \\
			&S_N^\prime(n) = \mathit{Running},
		\end{aligned} & \text{otherwise}
	\end{cases}
\end{equation}

In words, the function \(F_{In}\) takes the tokens from the input flows, and puts it in the given node, thus starting it's execution. The rule how it selects when is enabled, and which flows to empty is determined by the node's type:

\begin{itemize}
	\item if it is a \emph{Merge} node, then it must only forward one, and only one token,
	\item if it is an \emph{Action} node, then it must forward all tokens, along with their values into the associated pins,
	\item otherwise, it must forward all tokens.
\end{itemize}

This way we can ensure the behaviours of the simple 'LAS' nodes, as well as the 'IR' nodes.

\begin{equation}
	F_{Run} : N \rightarrow
	\begin{cases}
		\begin{aligned}[b]
			&V_N^\prime(n) = F_{Action}(n, V_N(n)) \\
			&S_N^\prime(n) = \mathit{Done},
		\end{aligned} & \text{if } n \in N_{Action} \\[8pt]
		S_N^\prime(n) = \mathit{Done}, & \text{otherwise}
	\end{cases}
\end{equation}

In words, the function \(F_{Run}\) stochastically sets the given node's state to \emph{Done}. If the given state is of type \emph{Action}, then it must have a deeper semantic associated with it, thus the associated \(F_{Action}\) function must also be called, which implements it's specific behaviour (more detail below).

\begin{equation}
	F_{Out} : N \rightarrow
	\begin{cases}
		\begin{aligned}[b]
			&\text{select } f_s \in \outflows{n} \wedge S_F(f_s) = \mathit{Empty} \wedge G(f_s) = \mathit{True} \\
			&V_F^\prime(f_s) = V_N(n) \\
			&S_F^\prime(f_s) = \mathit{Full} \\
			&S_N^\prime(n) = \mathit{Idle},
		\end{aligned} & \text{if } n \in N_\mathit{Decision} \\[8pt]
		\begin{aligned}[b]
			&V_F^\prime(f) = \mathit{PV}_N(n, \sourcepin{f}) : \forall f \in \outflows{n} \bigcap F_D \\
			&S_F^\prime(f) = \mathit{Full} : \forall f \in \outflows{n} \\
			&S_N^\prime(n) = \mathit{Idle},
		\end{aligned} & \text{if } n \in N_\mathit{Action} \\[8pt]
		\begin{aligned}[b]
			&S_F^\prime(f) = \mathit{Full} : \forall f \in \outflows{n} \\
			&S_N^\prime(n) = \mathit{Idle},
		\end{aligned} & \text{otherwise}
	\end{cases}
\end{equation}

The function \(F_{Our}\) takes the token from node, and puts it in its output flows, thus ending it's execution. The rule how it selects when is enabled, and which flows to fill is determined by the node's type:

\begin{itemize}
\item if it is a \emph{Decision} node, then it must only forward the token on one of its \emph{enabled} flows,
\item if it is an \emph{Action} node, then it must forward all tokens, along with their values from the associated pins,
\item otherwise, it must forward all tokens.
\end{itemize}

\end{definition}\label{def:activity-state-change}

\autoref{fig:activity-state-function} depicts these three functions, and how they change the state of a given node.

\fi

As the final step, we construct the transition \(T\), by applying the \(O_\mathit{Node}(n)\) function on each node:

\begin{equation*}
	T = \bigcup_{n \in N} O_\mathit{Node}(n)
\end{equation*}
