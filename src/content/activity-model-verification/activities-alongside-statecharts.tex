%----------------------------------------------------------------------------
\section{Activities Alongside Statecharts}
%----------------------------------------------------------------------------

fentebb leírtam, hogyan lehet leképezni proces modelleket, de így csak magukban futhatnak. Itt most leírom, hogy milyen nehézségeket okozhat, amikor egymás mellett futtatjuk őket:

do activity:

amikor a state aktív, futhat az aktivity, de ez pár problémával jön: párhuzamossági és szinkornizációs problémák

action:

egy activity értékét ki kell lapítani, hogy a tranzíció tüzelése után azonnal értelmezhető legyen (nem felezhetjük el az adott tranzíciót, mert akkor érvénytelen állapotaink lesznek)

unblock:

még nagy kérdés, hogy milyen szinten bontsuk fel a lépéseket, mennyire legyenek atomikusak, mik futhatnak egymás mellett, stb

alap elképzelésben akár egy activity akár egy állapotgép definiálhat egy komponsens viselkedését, de jelenleg ezt is egyszerűsített módon implementáltuk; az activity az állapotgép része lehet

megoldás a kérdésekre:

megoldásként a legegyszerűbb módszert választottuk, mert első sorban az a kérdés, hogy ez a módszer egyáltalán alkalmazható-e; a másik mindenképpen bonyolítja az implementációt és a létrejövő modellt is
