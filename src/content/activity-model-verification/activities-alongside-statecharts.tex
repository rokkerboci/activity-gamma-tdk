%----------------------------------------------------------------------------
\section{Activities Alongside Statecharts}\label{sec:activities-alongside-statecharts}
%----------------------------------------------------------------------------

In this section, I present the various questions that arose when I integrated GATL into Gamma statecharts, and present my solutions and reasoning for said problems.

\subsection{Calling Activities}

In order to create the connection between statecharts and activities, the user needs an interface to do so. There are many ways to call activities from statecharts:

\paragraph{Transition Actions}

The first solution that came in mind was calling activities from transition actions. At first glance, this makes the most sense, however, as activities are inherently parallel in nature, the resulting implementation would have to \emph{flatten} the activity model - the action cannot contain loops, and has to be atomic. Because of time constraints, this feature was not chosen. 

\paragraph{Do Behaviours}

In SysML, states may have do behaviours. This means, that the \emph{behaviour} (activity in our case) is \emph{ran} while the state machine is inside the given state, and \emph{halted} when the state is left. However, the semantics are not clear.

Semantically, the halting of the activity can happen in two ways: 

\begin{itemize}
	\item The state \emph{signals} the activity, that is should end, and waits until it halts - creating a synchronisation step.
	\item The activity checks if it should run - whether the state machine is in the given state.
\end{itemize}

Another open question is what happens, when a state has a loop transition into itself? Is the activity stopped, asked to stop, or a whole new activity is created?

For simplicities sake, I chose do behaviours with option two: check each time when the activity would do something, whether it is \emph{enabled} to run, or not. Additionally, when the containing state becomes active, the activity nodes are reset to \emph{Idle} state, and the initial nodes are started.

\subsection{Activities Defining Components}

In SysML, components' behaviour may be defined directly by activities. However in Gamma, all components use reactive semantics, which would have been difficult to implement. Thus, activities may only appear inside statecharts for now.

Although this constrains us to only use statecharts directly, we can still utilise the reactive nature of components, by adding a special activity node, that listens for such events.

\subsection{Trigger Node}

\emph{TriggerNodes} extend the activity semantics, by adding an additional assumption operation inside the \(	O_\mathit{NodeRun}\) operation function, checking if the given event has been received. The following example shows a trigger node, that will be executed only when it has a token from the initial node, and the \emph{start} event is received from the \emph{Control} port.
\vspace{6mm}
\begin{lstlisting}[language=statechart]
statechart Statechart [
	port connection : requires Control
] {
	// definitions ...
	
	activity DoActivity {
		initial Initial
		trigger AcceptEvent when connection.start
		final Final
		
		control flow from Initial to AcceptEvent
		control flow from AcceptEvent to Final
	}
}
\end{lstlisting}
