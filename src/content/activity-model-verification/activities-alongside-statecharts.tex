%----------------------------------------------------------------------------
\section{Activities Alongside Statecharts}\label{sec:activities-alongside-statecharts}
%----------------------------------------------------------------------------

In the real world, SysML models often contain state machines that call activity diagrams using a state's \emph{do behaviour}, as often times this makes the design process easier. However, Gamma does not support statecharts containing \emph{do activity} actions. Thus, first I extend the Gamma meta-model to support calling activities from states.

\subsubsection{Gamma Extensions}

First, I extended Gamma \emph{States} to include \emph{do} actions as well among \emph{entry} and \emph{exit} actions (see \autoref{fig:gamma-statechart-extension}). In order to enable calling activities from statecharts, I added the \emph{CallActivityAction}, which has a reference to an activity declaration (see \autoref{fig:gamma-statechart-extension}). These changes can be seen visually in \autoref{fig:gamma-activity-overview}. Added parts are outlined with green, while edited parts are outlined with orange.

\begin{figure}[!ht]
	\begin{subfigure}{.5\textwidth}
		\centering
		\includesvg[inkscapelatex=false, width=60mm, keepaspectratio]{gamma-activity-extension}
		\caption{Call activity action}
		\label{fig:gamma-activity-extension}
	\end{subfigure}%
	\begin{subfigure}{.5\textwidth}
		\centering
		\includesvg[inkscapelatex=false, width=40mm, keepaspectratio]{gamma-statechart-extension}
		\caption{State modification}
		\label{fig:gamma-statechart-extension}
	\end{subfigure}
	\caption{Extensions to the Gamma meta-model}
	\label{fig:gamma-extension}
\end{figure}

\begin{figure}[!ht]
	\centering
	\includesvg[inkscapelatex=false, width=120mm, keepaspectratio]{gamma-functionality-activity-overview}
	\caption{Overview of my modification to the Gamma framework.}
	\label{fig:gamma-activity-overview}
\end{figure}

\subsection{Challenges of Calling Activities}

how to integrate into gamma, action, do action, etc, transition
gamma használja az xsts-t -> xsts-re kell trafózni


ez a rész csak jegyzet

Initial/final node, action node, composite activity node, decision/merge node, fork/join node, pins, control/data flows.

Do behaviour: 

call activity action on transitions is not supported, would have to be inlined and flattened, atomic functions, interlaced with line execution semantics (future work)

- call activity context függő - hosszú (nem atomikus) dolgot nem lehet akárhol meghívni (transition, sima action, stb)
- validáció: inline környezetben tilos várakozás és ciklus (vagy csak x mélységig)
- sok köztes nem stabil állapot
- nagy változás a trafóban

ezért csak do activity van

fentebb leírtam, hogyan lehet leképezni proces modelleket, de így csak magukban futhatnak. Itt most leírom, hogy milyen nehézségeket okozhat, amikor egymás mellett futtatjuk őket:

do activity:

amikor a state aktív, futhat az aktivity, de ez pár problémával jön: párhuzamossági és szinkornizációs problémák

action:

egy activity értékét ki kell lapítani, hogy a tranzíció tüzelése után azonnal értelmezhető legyen (nem felezhetjük el az adott tranzíciót, mert akkor érvénytelen állapotaink lesznek)

unblock:

még nagy kérdés, hogy milyen szinten bontsuk fel a lépéseket, mennyire legyenek atomikusak, mik futhatnak egymás mellett, stb

alap elképzelésben akár egy activity akár egy állapotgép definiálhat egy komponsens viselkedését, de jelenleg ezt is egyszerűsített módon implementáltuk; az activity az állapotgép része lehet

megoldás a kérdésekre:

megoldásként a legegyszerűbb módszert választottuk, mert első sorban az a kérdés, hogy ez a módszer egyáltalán alkalmazható-e; a másik mindenképpen bonyolítja az implementációt és a létrejövő modellt is
