%----------------------------------------------------------------------------
\section{Integration Semantics}\label{sec:integration-semantics}
%----------------------------------------------------------------------------

As presented in the transformation pipeline (\autoref{fig:gamma-overview}), Gamma transforms the \emph{statecharts} and \emph{components} into an XSTS model instance. In the following, I show how to merge XSTS created from \emph{components} with the XSTS created from activities.

The unified transformation is carried out in three steps. \autoref{ssec:preprocess-components} presents how we preprocess the composite model by creating activity \emph{instances} and adding the initialisation action to the calling states. Next, in \autoref{ssec:transform-components} we transform the components using the Gamma transformation pipeline and then transform the called (now unique) activities. The resulting XSTS models are merged, resulting in the final XSTS. This XSTS model then can be forwarded to the model checkers.

\subsection{Preprocess Components}\label{ssec:preprocess-components}

As the first step, we find all states that contain a \emph{call activity} do action -- denoted as \( S \). For each \(s \in S\), we construct a new \emph{ActivityInstance} for the called activity \( S_\mathit{Act}\), and add an \emph{InitialiseActivityAction} to its entry actions. This step ensures that two states calling the same activity will not have conflicting variables in the resulting XSTS\footnote{In XSTS all variables are defined the a global scope, thus they must have unique name}, and the called activity is initialised correctly before it is started. \autoref{fig:gamma-lowlevel-extension} shows the extensions added to the Gamma metamodel.

\begin{figure}[!ht]
	\centering
	\includesvg[inkscapelatex=false, width=100mm, keepaspectratio]{gamma-lowlevel-extension}
	\caption{Activity instance and initialise activity extensions for the Gamma metamodel.}
	\label{fig:gamma-lowlevel-extension}
\end{figure}

\subsection{Transform Components and Activities}\label{ssec:transform-components}

As the next step, we transform the Gamma components, denoting the resulting model as \( \mathit{XSTS_{Comp}} \), and the activity instances, denoted as \( \mathit{XSTS_{S_{Act}}} \). In order to prevent the activity from running when the state is not active, an additional \emph{assume} operation is added to the \(O_\mathit{Node}(n)\) function, checking whether the associated state is active or not. Finally, the resulting XSTS models are merged.

\begin{equation*}
	XSTS = \mathit{XSTS_{Comp}} \bigcup \left( \bigcup_{s_\mathit{Act} \in S} \mathit{XSTS_{s_{Act}}} \right)
\end{equation*}
