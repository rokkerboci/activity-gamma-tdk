%----------------------------------------------------------------------------
\section{Activities as State-based Models}
%----------------------------------------------------------------------------



\subsubsection{Gamma Extension}

\begin{figure}[!ht]
	\begin{subfigure}{.5\textwidth}
		\centering
		\includesvg[inkscapelatex=false, width=70mm, keepaspectratio]{gamma-activity-extension}
		\caption{activity}
		\label{fig:gamma-activity-extension}
	\end{subfigure}%
	\begin{subfigure}{.5\textwidth}
		\centering
		\includesvg[inkscapelatex=false, width=70mm, keepaspectratio]{gamma-statechart-extension}
		\caption{statechart}
		\label{fig:gamma-statechart-extension}
	\end{subfigure}
	\begin{subfigure}{\textwidth}
		\centering
		\includesvg[inkscapelatex=false, width=70mm, keepaspectratio]{gamma-lowlevel-extension}
		\caption{lowlevel}
		\label{fig:gamma-lowlevel-extension}
	\end{subfigure}
	\caption{extension}
	\label{fig:gamma-extension}
\end{figure}

\begin{figure}[!ht]
	\centering
	\includesvg[inkscapelatex=false, width=120mm, keepaspectratio]{gamma-functionality-activity-overview}
	\caption{Overview of my modification to the Gamma framework. Added parts are outlined with green, while edited parts are outlined with orange.}
	\label{fig:gamma-activity-overview}
\end{figure}

formalizmus trafózása statechart mellett xsts-re. a végeredmény miért ekvivalens