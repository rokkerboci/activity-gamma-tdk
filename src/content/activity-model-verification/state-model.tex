%----------------------------------------------------------------------------
\section{Activities as State-based Models}
%----------------------------------------------------------------------------

activity vagy proces-model?

ahogy fentebb írtam, tokenek haladnak, és tranzíciók vannak.

A process-orionted model előnye itt a hátrányunk; mivel sok process mehet teljesen függetlenül egymástól, ezért nagyon nehéz szépen leírni őket

tegyük fel, hogy a node-ok és csatlakozók token-tartalma egy állapot; vagy van benne, vagy nincs. ehhez még hozzátesszük azt, hogy a node éppen fut, kész, vagy nem csinál semmit.

ezek alapján le tudunk írni bármilyen process-oriontált modelt állapot alapú modellben, feltételezve, hogy bármikor bármelyik tranzíció tüzelhet.

itt leírom az activity különböző node-jait, és azoknak a különböző szabályait (merge, choice, stb)

majd leírom a különböző csatlakozók segítségével is
