%----------------------------------------------------------------------------
\chapter{Evaluation}\label{ch:evaluation}
%----------------------------------------------------------------------------

In this chapter, I present two case studies to showcase the usability and verifiability of the Gamma Activity Language, using a handcrafted simple model and an industry example model.

\section{Case Study - Compilation}

The first case study is about the running example introduced in \autoref{fig:activity-running-example}. The goal of this experiment is to evaluate the formal verification capabilities of the language using a toy model.

\subsection{Modeling}

The GATL representation of the model has already been presented in \autoref{lst:gamma-activity}, however, some modifications have to be implemented in order to run the verification. As activities can not be verified directly, they must be wrapped within a statechart. Also, because GPL properties can not reference activity model elements, an additional flag (variable declaration of type boolean) must be added, to enable the construction of a reachability property specifying the end of the activity. The modifications can be seen in \autoref{lst:modified-compilation}.

\begin{lstlisting}[language=statechart, linewidth=0.80\textwidth, xleftmargin=0.20\textwidth, caption={The textual representation of the modified compilation example.}, label={lst:modified-compilation}]
statechart CompilationWrapper {
	var activityDone : boolean := false
	region Main {
		initial MainEntry
		state Wrapper {
			do / call CompilationProcess;
		}
	}
	transition from MainEntry to Wrapper
	activity CompilationProcess {
		// ...
		action Done : activity [language=action] {
			activityDone := true;
		}
		final Final
		// ...
		control flow from Decision to Done [!errors]
		control flow from Done to Final
	}
}
\end{lstlisting}

The next step was to wrap the statechart in a \emph{cascade} component, and define the reachability property of the variable \emph{activityDone} being true:

\begin{lstlisting}[language=statechart, linewidth=0.80\textwidth, xleftmargin=0.20\textwidth]
cascade Compilation {
	component CompilationWrapper : CompilationWrapper
}
\end{lstlisting}

\begin{lstlisting}[language=statechart, linewidth=0.80\textwidth, xleftmargin=0.20\textwidth]
component Compilation

// Can the variable activityDone ever be true?
E F [ { variable CompilationWrapper.activityDone } ] 
\end{lstlisting}

Finally, I ran the verification using the Theta and the UPPAAL model checkers. 

\subsection{Results and Conclusion}

\autoref{tab:cs1} contains the execution time of the two model checkers. Interestingly, Theta took -- on average -- one order of magnitude more time, then UPPAAL.

The trace \autoref{lst:cs1} shows the steps the model checkers took to reach the state described by the property (\emph{activityDone = true}). The first step shows the initial values of the system, while the following steps show how the values change after the component is scheduled. As can be seen, the second step is repeated \(17\) times before the variable becomes true. As there are no transitions in the XSTS model from the statechart, the only transitions that can be fired come from the activity. The token in the activity goes through eight\footnote{Merge, Fork, Read(1,2), Compile(1,2), Join, Link, Decision, Edit, Done} nodes and eight flows to reach node \emph{Done}, after which an additional step is needed to execute its action: thus resulting in a total of \( 8 + 8 + 1 = 17 \) steps.

\begin{table}[!ht]
	\centering
	\begin{tabular}{|c|c|c|}
		\hline
		\textbf{Model Checker} & \textbf{Successful Verification} & \textbf{Execution Time} \\ \hline
		Theta                  & \emph{Yes}                              & \(1332\) ms                 \\ \hline
		UPPAAL                 & \emph{Yes}                              & \(162\) ms					\\ \hline
	\end{tabular}
	\caption{The execution times for the compilation example.}
	\label{tab:cs1}
\end{table}

\begin{lstlisting}[float,language=statechart, linewidth=0.80\textwidth, xleftmargin=0.20\textwidth, caption={The trace of the run formal verification.}, label={lst:cs1}]
step {
	act {
		reset
	}
	assert {
		CompilationWrapper.Wrapper
		CompilationWrapper.activityDone = false
	}
}
step {
	act {
		schedule component
	}
	assert {
		CompilationWrapper.Wrapper
		CompilationWrapper.activityDone = false
	}
}
// 16 more
step {
	act {
		schedule component
	}
	assert {
		CompilationWrapper.Wrapper
		CompilationWrapper.activityDone = true
	}
}
\end{lstlisting}

The reason for the repetition is the lack of trace information regarding the activity model; the XSTS transitions are executed as expected, however, the trace mechanism of Gamma can not display these changes.

This case study showed that the Gamma Activity Language formalism does indeed work for simple examples. However, there is still more work to be done, to let the user choose activity nodes as reachability properties, and to show activity variables in the resulting trace (see \autoref{ch:conclusion}).

\section{Case Study - Simple Space Mission}

This section introduces an example model from the aerospace domain which we use as a case study to demonstrate the capability of the Gamma Activity Language to model complex SysML behavioural models. The example model was proposed by NASA in the context of the OpenMBEE\footnote{https://www.openmbee.org/} framework. The goal of OpenMBEE is to create a common model repository to facilitate tool integration, so the ability to handle models in the scope of this project can raise the relevance of any model analysis tool.

The Gamma models used in this section are a modification of a previous case study~\cite{mixed_statecharts_2020}, in which the activities were mapped to parallel regions and composite states. For this case study, I modified the models to use GATL to represent the activities.

\subsection{System Modeling}

\begin{figure}[]
	\centering
	\includesvg[inkscapelatex=false, width=80mm, keepaspectratio]{ground_station}
	\caption{The state machine describing the behaviour of the ground station component.}
	\label{fig:ground_station}
\end{figure}

\begin{figure}[]
	\centering
	\includesvg[inkscapelatex=false, width=100mm, keepaspectratio]{spacecraft_modes}
	\caption{The state machine describing the behaviour of the spacecraft component.}
	\label{fig:spacecraft_modes}
\end{figure}

\begin{figure}[]
	\centering
	\includesvg[inkscapelatex=false, width=90mm, keepaspectratio]{recharge_batteries}
	\caption{The activity diagram describing the battery recharge process of the spacecraft component.}
	\label{fig:recharge_batteries}
\end{figure}

The Simple Space Mission SysML model describes how a satellite with limited battery charge communicates with a ground station, where data transfer results in large power losses. The state-based behaviour of the system can be seen in Figures \ref{fig:ground_station} and \ref{fig:spacecraft_modes}, while the more complex activities are depicted in Figures \ref{fig:recharge_batteries} and \ref{fig:transmit_data}.

The challenge of the mapping is posed by the elements that are not supported in GATL. The SysML activities use \emph{duration constraints} to model how much time a given action takes to execute, along with \emph{interrupting edges} and \emph{internal signals} to simplify the models. The GATL language does not support these elements, however, they can be substituted with other modeling constructs. Duration constraints can be modeled using \emph{timeouts} and \emph{trigger nodes} to insert waiting before the given action can be executed -- although the time waited will be deterministic, unlike the SysML specification. The behaviour defined with the interrupting edge and internal signals can be substituted with additional boolean variables that signal when the execution of the activity should be halted. The easiest way to halt the activity is to leave its state. To detect the changes in the flags, I added a timeout that tests for the flags every second. The created Gamma models can be found in \autoref{sec:spacecraft-model}

\subsection{Results and Conclusion}

To check the conformance of the Gamma model with the original SysML model, I generated a state covering test set using the test generation functionality of Gamma, which produces reachability properties for all states in the system. \autoref{tab:cs2-uppaal} shows the execution times of the UPPAAL model checker. The table clearly shows how the state space of the low-level formalism ``exploded'' when checking for the \textsl{Satellite.Battery.Recharging} state, hence the high duration.

\begin{table}[!ht]
	\centering
	\begin{tabular}{|l|c|c|}
		\hline
		\textbf{Reachability Property}                       & \textbf{UPPAAL} & \textbf{Theta} \\ \hline
		Station.Main.Idle                    & 239 ms                              & 960 ms                  \\ \hline
		Station.Main.Operation               & 222 ms                              & 1 450 ms                  \\ \hline
		Satellite.Communication.WaitingPing  & 179 ms                              & 890 ms                  \\ \hline
		Satellite.Communication.Transmitting & 220 ms                              & 1 172 ms                  \\ \hline
		Satellite.Battery.NotRecharging      & 171 ms                              & 947 ms                  \\ \hline
		Satellite.Battery.Recharging         & 35 998 ms                              & \textit{out of memory}               \\ \hline
		Satellite.Battery.transmitionDone         & \textit{timeout}                              & 1 931 ms              \\ \hline
		Satellite.Battery.batteryRecharging         & 42 462 ms                              & \textit{timeout}               \\ \hline
		Satellite.Battery.batteryFullyCharged         & \textit{timeout}                              & 2 018 ms              \\ \hline
		Satellite.Battery.battery = 95         & 2 369 ms                             & \textit{timeout}               \\ \hline
		Satellite.Battery.battery = 90         & 9 899 ms                              & \textit{timeout}               \\ \hline
		Satellite.Battery.battery = 80         & 45 703 ms                              & \textit{timeout}               \\ \hline
		Satellite.Battery.battery = 70         & \textit{timeout}                              & \textit{timeout}               \\ \hline
	\end{tabular}
	\caption{The execution times for the Simple Space Mission system using UPPAAL.}
	\label{tab:cs2-uppaal}
\end{table}

In contrast, \autoref{tab:cs2-theta} shows the execution times of the Theta model checker. Every property took significantly more time to run on Theta, moreover, the previous state space explosion caused Theta to run out of memory.

This case study showed, that real world SysML models can be mapped to the Gamma Activity Language, although, some SysML elements may have to be transformed into a series of GATL elements manually. However, the model verification takes exponentially more time as we increase the model elements.

\begin{figure}[]
	\centering
	\includesvg[inkscapelatex=false, width=110mm, keepaspectratio]{transmit_data}
	\caption{The activity diagram describing the data transmission process of the spacecraft component.}
	\label{fig:transmit_data}
\end{figure}
