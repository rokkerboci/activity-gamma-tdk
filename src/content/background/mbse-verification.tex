%----------------------------------------------------------------------------
\section{Verifcation Techniques in Systems Engineering}\label{sec:mbse-verification}
%----------------------------------------------------------------------------

Using Model-based Systems Engineering helps reduce the amount of time needed to develop complicated systems, however, as all development processes, it is still prone to human error. In the section I introduce a typical engineering workflow based on the widely known V-model. This workflow is used as a general guideline in the development of safety-critical systems, however, many variants have been developed for the special needs of the different sectors. The V-model defines the elementary steps and draws a general workflow for the design and implementation of the system as depicted on Figure 1.2. In addition, the workflow defines verification and validation steps in the development to the correctness of the system. 

As you can see in \autoref{fig:v-model}, the workflow is split into multiple phases, all of which having their own \emph{design}, \emph{implementation} and \emph{verification} steps. The first phase is requirement design, where the requirements of the system are identified and collected. In the second phase, designing the architecture provides the necessary decomposition to be able to construct the component level design. At each phase, the designer refines the result of the former phases by providing more details. As the result of the final phase in the left wing, the implementation is produced for each design. The implementation has to be tested and verified against coding and other implementation errors. After the component level validation, system integration builds the smaller pieces together where extensive integration testing is executed to validate that the components work properly together. Finally, system validation ensures that the system is indeed what the customer desires.