%----------------------------------------------------------------------------
\section{Modeling Formalisms}\label{sec:formalisms}
%----------------------------------------------------------------------------

\subsection{Behaviour Formalisms}

\subsubsection{State-based}

ábra egy állapot alapú modellről

állapot alapú modellek állapotokra vannak osztva, melyek között átmenetek vannak

ezek az átmenetek rendelkeznek triggerrel, guarddal és action-nel

egy állapot lehet aktív régionként, és több régió is lehet

ezekről mind 1-1 ábra, ahogy váltanak a tranzíciók

\subsubsection{Process-oriented}

ábra egy folyamat-oriontált modellről

process csomópontok és csatlakozások vannak

ezeken tokenek futnak végig

adott csomópont akkor futhat, ha elég token juthat be a bemeneti csatlakozásokon

miután végzett a folyamat, 1-1 tokent ad ki a kimeneti csatlakozásokon

ezekről mind 1-1 ábra, ahogy haladnak a tokenek, stb

\subsubsection{Differences}

alapvető különbségek leírása, működés és használatban főleg

\subsection{SysML}

\subsubsection{Statechart}

\subsubsection{Activity}

\subsubsection{Combining Behaviours}