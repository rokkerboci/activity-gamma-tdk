%----------------------------------------------------------------------------
\section{Extended Symbolic Transition System}\label{sec:xsts}
%----------------------------------------------------------------------------

The high-level nature of engineering models means they are easy-to-use for engineers, but leads to difficulties during the formal verification process. SysML state machines and acitivty diagrams for example contain high-level modeling constructs that make the modeling workflow more intuitive and enable the modeling of complex systems, however they are difficult to process using formal methods that are defined on low-level mathematical formalism. In this section, I introduce the Extended Symbolic Transition System \cite{xsts} language, which is a low-level modeling formalism designed to bridge the abstraction gap between engineering models and formal methods.

\subsection{Formal definition}

\begin{definition}[Extended symbolic transition system]
	
	An \emph{Extended symbolic transition system} is a tuple \( XSTS = (D, V, V_C, IV, Tr, In, En) \), where:
	
	\begin{itemize}
		\item \(D = \{ D_{v_1}, D_{v_2}, \dots, D_{v_n} \} \) is a set of value domains;
		\item \(V = \{ v_1, v_2, \dots, v_n \} \) is a set of variables with domains \(D_{v_1}, D_{v_2}, \dots, D_{v_n}\);
		\item \(V_C \subseteq V\) is a set of variables marked as \emph{control variables};
		\item \(IV \in D_{v_1} \times D_{v_2} \times \dots \times D_{v_n}\) is the \emph{initial value function} used to describe the initial state. The initial value function \(IV\) assigns an initial value \(IV(v) \in D_v\) to variables \(v \in V\) of their domain \(D_v\);
		\item \(Tr \subseteq Ops\) is a set of operations, representing the \emph{internal transition relation}; it describes the internal behaviour of the system;
		\item \(In \subseteq Ops\) is a set of operations, representing the \emph{initialisation transition relation}; it is used to describe more complex initialisation, and is executed once and only once, at the very beginning;
		\item \(En \subseteq Ops\) is a set of operations, representing the \emph{environmental transition relation}; it is used to model the system's interactions with its environment.
	\end{itemize}
\end{definition}\label{def:xsts}

In any state of the system, a single operation is selected from the sets introduced above (\(Tr\), \(In\) and \(En\)). The set from where the operation can be selected depends on the current state: In the initial state - which is described by the initialization vector \(IV\) - only operations from the \(In\) set can be executed. Operations from the \(In\) set can only fire in the initial state and nowhere else. After that, \(En\) and \(Tr\) are selected in an alternating manner.

Operations \(op \in Ops\) describe the transitions between states of the system, where \(Ops\) is the set of all possible transitions. All operations are atomic in the sense that they are either executed in their entirety or none at all. XSTS defines the following operations:

\paragraph{Basic operations} 

Basic operations contain no inner (nested) operations.

\begin{itemize}
	\item \emph{Assignments} assign a given value \(v\) from domain \(D_n\) to variable \(V_n\).
	\item \emph{Havocs} behave likewise, except the value is not predetermined; giving a way to assign random value to a variable.
	\item Lastly, \emph{assume} operations check a condition, and can only be executed if their condition evaluates to \emph{true}.
\end{itemize}

\paragraph{Composite operations}

Composite operations contain other operations, and can be used to describe complex control stuctures. 

\begin{itemize}
	\item \emph{Sequences} are essentially multiple operations executed one after the other.
	\item \emph{Parallels} are like sequences, however they execute all operations at the same time.
	\item And lastly, \emph{choices} model non-deterministic choices between multiple operations; one and only one branch of the choice operation is selected for execution.
\end{itemize}

Note that while these are composite operations, their execution is still atomic; i.e., a potential false evaluation of a containing assume operation prevents the execution of all operations in that particular branch. E.g., if a sequence's 2nd operation is an assume operation, which cannot execute, then the whole sequence operation will be prevented from execution.

\subsection{Traffic Light Controller Example}

\autoref{lst:gamma-xsts} is the textual representation of the traffic light controller statechart (\autoref{fig:sysml_state_machine}) transformed to XSTS\footnote{The transformation was done using Gamma.}. For a more exact presentation please see \autoref{sec:xsts_language}.

\begin{lstlisting}[float,language=xsts, caption={Gamma XSTS Language representing the traffic light controller statechart.}, label={lst:gamma-xsts}]
type ActivityNodeState : { __Idle__, __Running__, __Done__ }
// ...
type Operating_Controller : { __Inactive__, Priority, Init, PriorityPrepares, Secondary, SecondaryPrepares }
var PoliceInterrupt_police_In_Controller : boolean = false
// ...
ctrl var main_region_Controller : Main_region_Controller = __Inactive__
ctrl var operating_Controller : Operating_Controller = __Inactive__
var SecondaryTimeout2_Controller : integer = 0

trans {
	// ...
	choice {
		assume (((main_region_Controller == Interrupted) && 
		// ...
		choice {
			assume (operating_Controller == Priority);
			SecondaryTimeout2_Controller := 0;
			PriorityControl_toggle_Out_Controller := true;
			SecondaryControl_toggle_Out_Controller := true;
			// ...
		}
	} or {
		// ...
	}
	PoliceInterrupt_police_In_Controller := false;
}
init {
	SecondaryTimeout2_Controller := (2 * 1000);
	main_region_Controller := __Inactive__;
	// ...
	PriorityPolice_police_Out_Controller := false;
	choice {
		assume (operating_Controller == __Inactive__);
		operating_Controller := Init;
	} or {
		assume (!((operating_Controller == __Inactive__)));
	}
	// ...
}
env {
	havoc PoliceInterrupt_police_In_Controller;
	PriorityPolice_police_Out_Controller := false;
	SecondaryControl_toggle_Out_Controller := false;
	PriorityControl_toggle_Out_Controller := false;
	SecondaryPolice_police_Out_Controller := false;
}
\end{lstlisting}
