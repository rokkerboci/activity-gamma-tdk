%----------------------------------------------------------------------------
\chapter{\bevezetes}
%----------------------------------------------------------------------------

Model-Based Systems Engineering is a widely used technique for designing, testing and executing otherwise too complicated systems in a well defined, \emph{high-level} language (e.g. SysML). Such systems usually come from critical domains (e.g. transportation, flight) which have a very low tolerance for errors. As such, early detection of otherwise deadly mistakes is mandatory.

Formal model verification is a technique (among many) for detecting such errors. They use formal proving algorithms to check whether a given system property (usually an undesired state) is \emph{reachable} and how to reach that state. However, these proving methods require low-level, \emph{state based mathematical models}, which are far from the high-level languages we are used to. In order to verify such models, we need to \emph{bridge} the gap between the two worlds.

Gamma is one of such frameworks. Right now Gamma only supports composite systems of only statecharts, however, real world models use activity diagrams heavily. For this reason, I propose to add an activity language to Gamma, and with it enable the systematic verification of a larger subset of SysML models.

TODO létező megoldások keresése és leírása (Beni pl)

TODO dolgozat felépítése
