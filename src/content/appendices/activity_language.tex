%----------------------------------------------------------------------------
\clearpage\section{Gamma Activity Language}\label{sec:gamma-activity-language-constructs}
%----------------------------------------------------------------------------

\subsection{Language elements}

The following section gives high level introduction into the syntax of the Gamma Activity DSL.

\subsubsection{Pins}\label{sssec:pins}

Pins can be declared the following way, where \verb|<direction>| can be either \verb|in| or \verb|out|, and \verb|type| a valid Gamma Expression type.

\begin{lstlisting}[language=activity]
<direction> <name> : <type>
\end{lstlisting}

For example: 

\begin{lstlisting}[language=activity]
in examplePin : integer
\end{lstlisting}

where the direction is \verb|in|, the name is \verb|examplePin| and the type is \verb|integer|.

\subsubsection*{Nodes}\label{sssec:nodes}

Nodes can be declared by stating the \verb|type| of the node and then it's name. The type determines the underlying meta element.

The available node types:

\begin{lstlisting}[language=activity]
initial InitialNode
decision DecisionNode
merge MergeNode
fork ForkNode
join JoinNode
final FinalNode
action ActionNode
\end{lstlisting}

\subsubsection{Flows}\label{sssec:flows}

The behaviour of the Activity can be described by stating \verb|data| or \verb|control| flows between two nodes. Flows may have \verb|guard|s on them, which limits when the flow can fire. Activities may only be from the current activity definition's children. Pins can be accessed using the \verb|.| accessor operator, the activity on the left hand side, and the pin's name on the right. The enclosing activity's name is \verb|self|.

Flows can be declared the following way, where \verb|<kind>| is the kind of flow, \verb|<source>| and \verb|<target>| is the source/target node or pin, and \verb|<guard>| is a Gamma Expression returning boolean:

\begin{lstlisting}[language=activity]
<kind> flow from <source> to <target> [<guard>]

control flow from activity1 to activity2
control flow from activity1 to activity2 [x == 10]
data flow from activity1.pin1 to activity2.pin2
data flow from self.pin to activity3.pin2
\end{lstlisting}

\subsubsection{Declarations}

Activity declarations state the \emph{name} of the activity, as well as its \emph{pins} \ref{sssec:pins}. 

\begin{lstlisting}[language=activity]
activity Example (
	//..pins..
) {
	//..body..
}
\end{lstlisting}

You can also declare activities inline by using the \verb|:| operator:

\begin{lstlisting}[language=activity]
activity Example {
	action InlineActivityExample : activity
}
\end{lstlisting}

\subsubsection{Definitions}

Activities also have definitions, which give them bodies. The body language can be either \verb|activity| or \verb|action| depending on the language metadata set. Using the \verb|action| language let's you use any Gamma Action expression, including timeout resetting, raising events through component ports, or simple arithmetic operations.

An example activity defined by an action body:

\begin{lstlisting}[language=activity]
activity Example (
	in x : integer,
	out y : integer
) [language=action] {
	self.y := self.x * 2;
}
\end{lstlisting}

Inline activities may also have pins and be defined using action language:

\begin{lstlisting}[language=activity]
activity Example {
	action InlineActivityExample : activity (
		in input : integer,
		out output : integer
	) [language=action] {
		self.output := self.input;
	}
}
\end{lstlisting}

A composite activity can be easily created using inline activity definition:

\begin{lstlisting}[language=activity]
activity Example {
	initial Init1
	final Final1
	
	action InlineActivityExample : activity {
		initial Init2
		final Final2
		
		control flow from Init2 to Final2
	}

	control flow from Init1 to InlineActivityExample
	control flow from InlineActivityExample to Final1
}
\end{lstlisting}

\subsubsection{Example}

A simple example activity can be seen in \autoref{lst:gamma-activity-adder}. This example shows two read actions, that return a random value, which are added together and logged to the console. The addition is defined using a \emph{NamedActivitDeclaraionReference}.

\lstinputlisting[float,language=activity, caption={Basic adder example.}, label={lst:gamma-activity-adder}]{code/language/add-example.gcd}
