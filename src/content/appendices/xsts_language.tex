%----------------------------------------------------------------------------
\section{XSTS Language}\label{sec:xsts_language}
%----------------------------------------------------------------------------

In this appendix I introduce the exact language constructs for the XSTS language 

\subsection*{Types}

XSTS contains two default variable types, logical variables (\emph{boolean}) and mathematical integers (\emph{integer}). XSTS also allows the user to define \emph{custom types}, similarly to enum types in common programming languages.

A custom type can be declared the following way:

\begin{lstlisting}[language=xsts]
type <name> : { <literal_1>, . . . , <literal_n> }

type Color : { RED, GREEN, BLUE }
\end{lstlisting}

\subsection*{Variables}

Variables can be declared the following way, where \verb|<value>| denotes the value that will be assigned to the variable in the initialization vector:

\begin{lstlisting}[language=xsts]
var <name> : <type> = <value>
\end{lstlisting}

The variable can only take a value of the specified type.

If the user wishes to declare a variable without an initial value, this is possible as well:

\begin{lstlisting}[language=xsts]
var <name> : <type>
\end{lstlisting}

A variable can be tagged as a control variable with the keyword \verb|ctrl|:

\begin{lstlisting}[language=xsts]
ctrl var <name> : <type>
\end{lstlisting}

In which case the variable \(v\) will also be added to \(V_C\) (the set of control variables).

Examples:

\begin{lstlisting}[language=xsts]
var a : integer
ctrl var b : boolean = false
var c : Color = RED
\end{lstlisting}

\subsubsection*{Operations}

The behaviour of XSTS can be described using basic and composite operations. Basic operations include assignments, assumptions and havocs. In the following, you can see an example for each, where \verb*|<expr>| is an expression returning a value, and \verb*|<varname>| is the name of a variable.

\begin{lstlisting}[language=xsts]
assume <expr>

<varname> := <expr>

havoc <varname>
\end{lstlisting}

Composite operations are either non-deterministic choices, sequences or parallels. Nondeterministic choices have the following syntax, where \verb|<operation>| are arbitrary basic or composite operations:


\begin{lstlisting}[language=xsts]
choice { 
	<operation>
} or { 
	<operation>
}
\end{lstlisting}

Sequences have the following syntax:

\begin{lstlisting}[language=xsts]
<operation>
<operation>
<operation>
\end{lstlisting}

And parallels have the following syntax:

\begin{lstlisting}[language=xsts]
parallel { 
	<operation>
	<operation>
}
\end{lstlisting}

\subsubsection*{Transitions}

Each transition is a single operation (basic or composite). We distinguish between three
sets of transitions, \emph{Tran}, \emph{Init} and \emph{Env}. Transitions are described with the following syntax, where \verb|<transition-set>| is either \verb|tran|, \verb|env| or \verb|init|:

\begin{lstlisting}[language=xsts]
<transition-set> {
	<operation>
} or {
	<operation>
} or
//...
or {
	<operation>
}
\end{lstlisting}