\pagenumbering{roman}
\setcounter{page}{1}

\selecthungarian

%----------------------------------------------------------------------------
% Abstract in Hungarian
%----------------------------------------------------------------------------
\chapter*{Kivonat}\addcontentsline{toc}{chapter}{Kivonat}

A biztonságkritikus rendszerek komplexitása folyamatosan növekedett az elmúlt években. A komplexitás csökkentése érdekében a modellalapú paradigma vált a meghatározó módszerré ilyen rendszerek tervezéshez. Modellalapú rendszertervezés során a komponensek viselkedését általában állapotalapú, vagy folyamatorientált modellek segítségével írjuk le. Az előbbi formalizmusa azt írja le, hogy a komponens milyen állapotokban lehet, míg az utóbbié azt, hogy milyen lépéseket hajthat végre, valamint milyen sorrendben. Gyakran ezen modellek valamilyen kombinálása a legjobb módja egy komplex komponens viselkedésének leírásához. 

Formális szemantikával rendelkező modellezési nyelvek lehetővé teszik a leírt viselkedés (kimerítő) verifikációját. Formális verifikáció használatával már a fejlesztés korai fázisaiban felfedezhetőek a hibák: a módszer ellenőrzi, hogy a rendszer egy adott (hibás) állapota elérhető-e, és amennyiben elérhető, ad hozzá egy elérési útvonalat. A formális verifikációs eszközök emiatt gyakran csak alacsony szintű, állapotalapú modelleken működnek, melyek messze vannak az emberek által könnyen érthető nyelvektől. Ezért, hogy magas szintű viselkedési modelleket tudjunk verifikálni, implementálnunk kell egy olyan modell transzformációt, mely megtartja a folyamat- és állapotalapú modellek szemantikáját azok kombinációja után is. 

Ebben a dolgozatban megvizsgálom a folyamatalapú modellek szemantikáját, valamint a kapcsolatukat egyéb hagyományos állapotalapú modellekkel. Emellett megoldásokat vetek fel a potenciális konfliktusokra a kombinált alacsonyszintű modellben. Munkám során a Gamma állapotgép kompozíciós keretrendszerre építek, mellyel komponensalapú reaktív rendszereket modellezhetünk és verifikálhatunk. Mivel a Gamma még nem támogatja az aktivitásokat, bevezetek egy új aktivitás nyelvet, melyhez a SysMLv2 szolgál inspirációként. Ezzel együtt implementálom hozzá a szükséges transzformációkat a Gamma alacsony szintű analízis formalizmusára. Végezetül pedig kiértékelem a koncepcionális és gyakorlati eredményeket esettanulmányokon és méréseken keresztül, valamint felvetek lehetséges fejlesztéseket és alkalmazásokat. 


\vfill
\selectenglish


%----------------------------------------------------------------------------
% Abstract in English
%----------------------------------------------------------------------------
\chapter*{Abstract}\addcontentsline{toc}{chapter}{Abstract}

The complexity of safety-critical systems has been increasing rapidly in recent years. To mitigate said complexity, the model-based paradigm has become the decisive way to design such systems. In model-based systems engineering, we usually define the behaviour of system components using state-based or process-oriented models. The former formalism describes what states the component can be in, while the latter describes what steps it can perform and in what order. Oftentimes, the best way to model the behaviour of a complex component is to combine these models in some way. 

Modelling languages with formal semantics enable the (exhaustive) verification of the described behaviour. Formal verification may be used to detect errors early during development by checking if a given (erroneous) state of the system can be reached, and if so, providing a way to reach it. Formal verification tools often require low-level state-based mathematical models, which are far from human-understandable languages. Thus, to enable the verification of high-level behavioural models, a model transformation must be implemented that preserves the semantics of both process-oriented and state-based models, even when combined.  

In this report, I analyse the semantics of process-oriented models, as well as their relation to traditional state-based models, and propose solutions for the possible conflicts in a combined low-level model. In my work, I build on the Gamma Statechart Composition Framework, which is a tool for modelling and verifying component-based reactive systems based on statecharts. Since Gamma does not support activities yet, I introduce a new activity language inspired by SysMLv2 and implement the necessary transformations to Gamma’s low-level analysis formalism. Finally, I evaluate the conceptual and practical results through case studies and measurements then propose potential improvements and applications. 

\vfill
\cleardoublepage

\selectthesislanguage

\newcounter{romanPage}
\setcounter{romanPage}{\value{page}}
\stepcounter{romanPage}